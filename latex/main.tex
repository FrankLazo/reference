\documentclass{article}     % Tipo de documento
\usepackage[utf8]{inputenc} % Paquete de codificación
\usepackage[spanish]{babel} % Paquete de diccionario
\usepackage{amsmath}        % Paquete para ecuaciones
\usepackage{lipsum}         % Paquete para texto aleatorio
\usepackage{multicol}       % Paquete para varias columnas
\usepackage{paracol}
\usepackage{xcolor}
\usepackage{graphicx}
\usepackage{wrapfig}
\usepackage{slashbox}
\usepackage{colortbl}
\usepackage{xcolor}[table]

\usepackage{xcolor}

\definecolor{Verde}{HTML}{88CC33}
\usepackage{xstring}

\newcommand{\Fourier}[1][eq]{
    \IfEqCase {#1}{
        {eq}{
            \begin{equation}
                \hat{f}(\xi) = \int_{-\infty}^{+\infty}f(x)e^{-2\pi ix\xi}dx
            \end{equation}
        }
        {disp}{
            \[
                \hat{f}(\xi) = \int_{-\infty}^{+\infty}f(x)e^{-2\pi ix\xi}dx
            \]
        }
        {inLine}{
            $\hat{f}(\xi) = \int_{-\infty}^{+\infty}f(x)e^{-2\pi ix\xi}dx$
        }
    }
}

\newcommand{\cuadratica}{
    \begin{equation}
        x_{1,2} = \frac{1}{2a}\Bigl(-b\pm \sqrt{b^2 - 4ac}\Bigr)
    \end{equation}
}

\renewcommand{\contentsname}{Tabla de contenido}
\renewcommand{\listfigurename}{Lista de figuras}

\graphicspath{{./images/}}

\newtheorem{theorem}{Teorema}

\title{Hola Mundo!}     % Título del documento
\author{Frank Lazo}     % Autor del documento
\date{\today}           % Fecha del documento

\begin{document}        % Inicio del documento

    \renewcommand{\figurename}{Fig.}
    \renewcommand{\listtablename}{Tablas}

    \maketitle          % Escribe el título, autor y fecha
    
    \tableofcontents
    \listoffigures
    \listoftables

    \begin{abstract}    % Ambiente para el resumen
        \lipsum[1]
    \end{abstract}
    
    \section*{Introducción}  % Inicia una sección del documento

        Hola Mundo!         % Contenido del documento

        \subsection{Título de la subsección}
            \lipsum[7]
    \section{Título de la sección}
        \lipsum[1]
        \subsection{Título de la subsección}
            \lipsum[2-3]
            \subsubsection{Título de la subsubsección}
                \lipsum[4]
                \paragraph{Título del párrafo} % Para títulos sin numeración
                    \lipsum[5]
    \section*{Título de la sección}
        \lipsum[1]
        \subsection*{Título de la subsección}
            \lipsum[2-3]
            \subsubsection*{Título de la subsubsección}
                \lipsum[4]
    \section{Título de la sección}
        \lipsum[1]
    
    \twocolumn
    \section{Título de la sección}
        \lipsum[1-5]

    \onecolumn
    \section{Título de la sección}
        \lipsum[1]
    
    %\renewcommand{\columnseprule}{.5pt}
    \setlength\columnsep{7.5mm}
    \begin{multicols}{3}
        \lipsum[2-3]
    \end{multicols}

    \begin{paracol}{2}[\section{Título muy largo de sección}]
        \lipsum[4]
        \switchcolumn
        Esta es la traducción del texto original.
        \switchcolumn*
        \lipsum[4]
        \switchcolumn
        \lipsum[3]
    \end{paracol}

    \section{Columnas de diferentes anchos}
    \columnratio{.6}
    \columncolor{red}
    \begin{paracol}{2}
        \lipsum[1]
        \switchcolumn
        \lipsum[4]
        \textcolor{green}{\lipsum[4]}
    \end{paracol}

    \section{Expresiones Matemáticas}
    El espacio se ajusta a la línea del texto: $\sum_{i = 1}^na_i$
    El espacio se ajusta a la expresión: $\displaystyle{\sum_{i = 1}^na_i}$

    La combinación de $n$ elementos de $r$ en $r$ se define:
    \[
    \binom{n}{r} = \frac{n!}{r!(n - r)}
    \]

    El número de permutaciones de $n$ en $r$ elementos es:
    \begin{equation}
    _nP_r = \frac{n!}{(n - r)!}
    \end{equation}

    Expresión con el ambiente equation sin numeración:
    \begin{equation*}
    p(x) = \frac{Q(x)}{P(x)},\; q(x) = \frac{R(x)}{P(x)}
    \end{equation*}

    \begin{equation*}
    E = \cfrac{1}{1 + \cfrac{1}{1 + \cfrac{1}{\ddots}}}
    \end{equation*}

    \[
        W(f_1, f_2) =
        \begin{vmatrix}
            x^2 & x|x| \\
            2x & \dfrac{2x^2}{|x|}
        \end{vmatrix}
    \]

    \[
        \left[
        \begin{array}{lcr}
            -0.1 & a & 0.1 \\
            -0.01 & a + 1 & 0.01
        \end{array}
        \right]
    \]

    \[
        \begin{split}
            L[c_1y_1 + c_2y_2] &= c_2L[y_1'' + py_1']\cdots \\
            &= c_2L[y_1] + c_2L[y_2]
        \end{split}
    \]

    \begin{gather}
        W' + p(x)W = 0 \\
        W(x) = Ce^{-\int p(x)dx} \notag \\
        E = mc^2
    \end{gather}

    \begin{theorem}
        Este es un teorema.
    \end{theorem}

    \textcolor{Verde}{Texto en color verde.}

    \cuadratica

    \Fourier[inLine]

    \LaTeX

    \section{Elementos flotantes}

    Elementos flotantes:
    \begin{figure}[ht]
        \centering
        \includegraphics[scale = .4]{images/portada.jpg}
        \caption{Leyenda de imagen}
    \end{figure}

    \begin{figure}[ht]
        \centering
        \includegraphics[scale = .3]{portada}
        \caption{Leyenda de imagen}
    \end{figure}

    \begin{figure}[ht]
        \begin{tabular}{|c|c|}
            \hline
            celda 11 & celda 12 \\
            \hline
            celda 21 & celda 22 \\
            \hline
        \end{tabular}
        \caption{Leyenda de tabla}
    \end{figure}

    \renewcommand{\tabcolsep}{20pt}
    \renewcommand{\arraystretch}{2}
    \renewcommand{\arrayrulewidth}{1pt}
    
    \lipsum[1]
    
    \begin{wraptable}{L}{0.8\textwidth}
        \begin{tabular}{|l|c|}
            \hline
            \backslashbox{celda 11a}{celda 11b} & celda 12 \\
            \hline
            celda 21 & celda 22 \\
            \hline
        \end{tabular}
        \caption{Leyenda de tabla}
    \end{wraptable}
    
    \lipsum[1]

    \renewcommand{\arrayrulewidth}{1pt}
    \begin{table}{ht}
        \begin{tabular}{|>{\columncolor{red!50}}ccc|}
            \hline
            \rowcolor{blue!50}
            celda 11 & celda 12 & celda 13 \\
            \hline
            celda 21 & celda 22 & celda 23 \\
            celda 31 & celda 32 & celda 33 \\
            celda 41 & celda 42 & celda 43 \\
            \hline
        \end{tabular}
        \caption{Leyenda de tabla}
    \end{table}

    \begin{table}{ht}
        \centering
        \rowcolors{2}{red!50}{green!50}
        \begin{tabular}{|>{\columncolor{red!50}}ccc|}
            \hline
            celda 11 & celda 12 & celda 13 \\
            \hline
            celda 21 & celda 22 & celda 23 \\
            celda 31 & celda 32 & celda 33 \\
            celda 41 & celda 42 & celda 43 \\
            \hline
        \end{tabular}
        \caption{Leyenda de tabla}
    \end{table}

\end{document}          % Fin del documento